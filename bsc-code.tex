% Error exponents for compound BSC
% Aditya Mahajan
% February 17, 2010

This article contains R source code used in the following paper

\startframedtext
  Aditya Mahajan and Sekhar Tatikonda, \quotation{Opportunistic capacity 
  and error exponent regions for compound channels with feedback}, 
  submitted to IEEE Transactions on Information Theory, 2010
\stopframedtext

This code is in written in R using the Sweave format. To process the file run

    Sweave("bsc-code.Rnw")




We consider a compound channel consisting of two BSCs with complementary
crossover probabilities, $p$ and $(1-p)$, where $0 < p < 1/2$ and $p$ is known
to the transmitter and the receiver. We denote this compound channel by

$$\ALPHABET Q_p \DEFINED \{\BSC_p, \BSC_{1-p}\} $$

where $\BSC_p$ denotes a binary symmetric channel with crossover probability
$p$. For convenience, we will index all variables by $p$ and $(1-p)$ rather than
by $1$ and $2$. For binary symmetric channel, the capacity and $B_Q$ term of
Burnashev exponent are given by

$$C_p = C_{1-p} = 1 - h(p)$$

\begin{Schunk}
\begin{Sinput}
> bsc.C <- function(p) 
> {
>   return (1 - binary.h(p)) 
> }
\end{Sinput}
\end{Schunk}

and
$$B_p = B_{1-p} = D(p \| 1-p) $$

\begin{Schunk}
\begin{Sinput}
> bsc.B <- function(p) 
> {
>   return (binary.D(p,1-p))
> }
\end{Sinput}
\end{Schunk}
  
where $h(p) = -p \log p - (1-p) \log (1-p)$ is the binary entropy function 

\begin{Schunk}
\begin{Sinput}
> binary.h <- function(p)
> {
>   return ( -p*log2(p) - (1-p)*log2(1-p))
> }
\end{Sinput}
\end{Schunk}

and
$D(p\|q) = -p \log (p/q) - (1-p) \log ( (1-p)/(1-q))$ is the binary
Kullback-Leibler function.

\begin{Schunk}
\begin{Sinput}
> binary.D <- function(x,y)
> {
>   return ( x*log2(x/y) + (1-x)*log2((1-x)/(1-y)))
> }
\end{Sinput}
\end{Schunk}

We choose the all zero sequence as a training sequence and estimate the channel
based on the type of the output sequence. If the empirical frequency of ones in
the output is less than $q$, $p < q < 1-p$, the channel is estimated as
$\BSC_p$; otherwise the channel is estimated as $\BSC_{1-p}$. For this class of
channel estimation rules, the estimation error probability is bounded by the
tail of the probability of the sum of independent random variables. 
From Hoeffding's inequality, the exponents of
the estimation errors are given by

$$T_p = D(q \| p), \quad T_{1-p} = D(q \| 1-p).$$


Suppose we want to communicate at rate $(R_p, R_{1-p})$, $R_p < C_p$
and $R_{1-p} < C_{1-p}$, using the coding scheme of the paper.
The lower bound of Proposition 2 simplifies to

$$E_p ≥ \frac{D(q \| p)   D(p \| 1-p)}
            {D(q\|p) + D(p \| 1-p)} 
            (1-γ_p)                         $$
and

$$E_{1-p} ≥ \frac{D(q \| 1-p) D(p \| 1-p)}
               {D(q\|1-p) + D(p \| 1-p)} 
                (1-γ_{1-p})                 $$

where $γ_p = R_p/C_p$ and $γ_{1-p} = R_{1-p}/C_{1-p}$. 

Now, we want to choose $q$ such that $E_p = E_{1-p}$ which is equivalent to
choosing $q$ such 

$$φ(q,p) = \frac{ (1-γ_p) }{ (1-γ_{1-p}) } $$

where

$$φ(q,p) = \frac{1 + D(p \| 1-p)/D(q\|p)}{1 + D(p \| 1-p)/D(q\|1-p) } $$

\begin{Schunk}
\begin{Sinput}
> compoundBsc.phi <- function(p,q)
> {
>   num = 1 + binary.D(p, 1-p) / binary.D(q, p)
>   den = 1 + binary.D(p, 1-p) / binary.D(q, 1-p)
>   return (num/den)
> }
\end{Sinput}
\end{Schunk}

This means that $q=0.5$, which maximally distinguishes
between $\BSC_p$ and $\BSC_{1-p}$ is optimal only when $γ_p =
γ_{1-p}$. For other values of $γ_p$ and $γ_{1-p}$, we need to
invert $φ(q,p)$ to determine the value of $q$. 

\begin{Schunk}
\begin{Sinput}
> compoundBsc.E <- function(p, γ_1, γ_2) {
>   # if (0 > γ_1 || 0 > γ_2 || 1 < γ_1 || 1 < γ_2)
>   #   stop("γ out of bound")
> 
>   exponent <- function (p,q,γ) 
>   {
>     num = binary.D (q,p) * binary.D (p, 1-p) * (1 - γ)
>     den = binary.D (q,p) + binary.D (p, 1-p)
>     return (num/den)
>   }
>   findQ <- function (q)
>   {
>     return (exponent(p,q,γ_1) - exponent(1-p,q,γ_2))
>   }
>   eps = 10e-3
>   q = uniroot(findQ, upper = 1-p - eps, lower = p + eps)$root
>   # if (abs (exponent(p,q,γ_1) - exponent(1-p,q,γ_2)) > eps )
>   #   warning(sprintf("q not within %f accuracy for p=%f, γ_1=%f, γ_2=%f",
>   #                   eps, p, γ_1, γ_2 ))
>   return (list(exp=exponent(p,q,γ_1), q=q))
> }
\end{Sinput}
\end{Schunk}

The code below gives the plot of $φ(q,0.1)$ for different values of $q$.

\begin{Schunk}
\begin{Sinput}
> # The affect of varying threshold q
> compoundBsc.plotPhi <- function (p=0.1)
> {
>   eps = 10e-3
>   q   = seq(p+eps, 1-p-eps, by=0.001)            
>   lab = parse(text=sprintf("varphi(italic(q), %.1f)", p))
> 
>   pdf ("threshold.pdf", width = 7.5, height = 4.5)  # Output file
>   par (lty = "solid", lwd = 1, bty = "l")           # An "L" box
>   plot(compoundBsc.phi(p, q), q,
>        lty  = "solid",
>        lwd  = 1      ,
>        type = "l"    ,
>        log  = "x"    ,
>        xaxt = "n"    ,
>        ylab = expression(italic(q)) ,
>        xlab = lab    )
> 
>   abline(v = 1,  lty = "1343")
>   abline(h = 0.5,lty = "1343")
> 
>   # The axis are fine tuned to p=0.1
>   axis(2,at=seq(p,1-p, by = 0.1))
> 
>   axis(1,at=c(1e-3,1e-2,1e-1,1,1e1,1e2,1e3,1e4), 
>   labels=c(
>              expression(10^-3), 
>              expression(10^-2), 
>              expression(10^-1), 
>              "1"              , 
>              "10"             , 
>              expression(10^2) , 
>              expression(10^3) ,
>              expression(10^4)))
>   dev.off()
>   embedFonts("threshold.pdf", options="-dPDFSETTINGS=/prepress")
> }
\end{Sinput}
\end{Schunk}


![plot][plot]



For different values of $γ_p$ and $γ_{1-p}$, the optimal value of $q$ is given
by the function below.

\begin{Schunk}
\begin{Sinput}
> # Show the table of q and E values
> compoundBsc.showValues <- function(p = 0.1,
>                                    γ_1 = rep(0.5,9), 
>                                    γ_2 = seq(0.1, 0.9, length=9) )
> {
>   len    = min(length(γ_1), length(γ_2))
>   output = data.frame(γ_1 = γ_1,γ_2 = γ_2,  q = rep(NA, len), exp = rep(NA, len))
>   for (i in 1:len)
>   {
>     values        = compoundBsc.E(p, γ_1[i], γ_2[i])
>     output$q[i]   = values$q 
>     output$exp[i] = values$exp
>   }
>   return (format(output, width=8, digits=4))
> }
\end{Sinput}
\end{Schunk}

\begin{Schunk}
\begin{Soutput}
       γ_1      γ_2        q      exp
1      0.5      0.1   0.5861  0.36663
2      0.5      0.2   0.5695  0.35112
3      0.5      0.3   0.5501  0.33295
4      0.5      0.4   0.5273  0.31142
5      0.5      0.5   0.5000  0.28551
6      0.5      0.6   0.4666  0.25373
7      0.5      0.7   0.4247  0.21391
8      0.5      0.8   0.3698  0.16284
9      0.5      0.9   0.2918  0.09526
\end{Soutput}
\end{Schunk}

The next function gives the error exponent for different values of 
$γ_p$ and $γ_{1-p}$.


\begin{Schunk}
\begin{Sinput}
> # Error exponents for different values of γ
> compoundBsc.plotExp <- function(p=0.1)
> {
>   library (lattice) 
>   p = 0.1
>   eps = 10e-3
>   γ_1 = seq(eps,1-eps,length=20)
>   γ_2 = seq(eps,1-eps,length=20)
> 
>   exponent <- function(g1, g2)
>   {
>     return(compoundBsc.showValues(p,g1,g2)$exp)
>   }
> 
>   z = outer(γ_1, γ_2, exponent) 
> 
>   pdf ("exponent.pdf", width = 9, height = 4.5)  # Output file
> 
>   points = expand.grid(x=γ_1, y=γ_2)
>   points$z = as.numeric(exponent(points$x, points$y))
> 
>   # Remove the box around the wireframe
>   trellis.par.set("axis.line", list(col = "transparent"))
>   plot1 <- wireframe(z ~ x * y, data=points, 
>             screen = list (z = -60, x = -75, y=-5), 
>             xlab = expression(gamma[italic(p)]),
>             ylab = expression(gamma[1-italic(p)]),
>             zlab = expression(italic(E)[italic(p)]),
>             scales = list(cex  = 0.75, col = "black", arrows=FALSE), 
>             shades = TRUE, colorkey=FALSE,  
>            )
>   plot(plot1, split = c(1, 1, 2, 1))
> 
>   trellis.par.set("axis.line", list(col = "black"))
>   plot2 <- contourplot(z ~ x * y, data=points, 
>             screen = list (z = -60, x = -75, y=-5), 
>             xlab = expression(gamma[italic(p)]),
>             ylab = expression(gamma[1-italic(p)]),
>             zlab = expression(italic(E)[italic(p)]),
>             cuts = 10, aspect = "fill", 
>             scales = list(cex  = 0.75, col  = "black", arrows=FALSE), 
>             region = FALSE, colorkey=FALSE,  pretty=TRUE,
>            )
> 
>   plot(plot2, split = c(2, 1, 2, 1), newpage = FALSE)
> }
\end{Sinput}
\end{Schunk}

\begin{Schunk}
\begin{Soutput}
null device 
          1 
\end{Soutput}
\end{Schunk}


![values][values]


The best possible error exponent for the compound channel $\ALPHABET Q_p$ is
not known. Nonetheless, we can compare the error exponent of our scheme with two
simple upper bounds (We only evaluate the bounds for $\BSC_p$. The bounds for
$\BSC_{1-p}$ are symmetric). The first upper bound on the error exponent is
given by the error exponent when both transmitter and receiver know that the
channel is $\BSC_p$. That, the first upper bound $E'_p$ equals to the Burnashev
exponent of $\BSC_p$, $B_p (1-γ_p)$. 

\begin{Schunk}
\begin{Sinput}
>   compound.burnashev <- function(p,R)
>   {
>     γ = R/bsc.C(p)
>     return (bsc.B(p) * (1-γ))
>   }
\end{Sinput}
\end{Schunk}

Since decreasing the rate of transmission cannot decrease the error exponent,
the second upper bound on the error exponent is given by the error exponent of
communicating at zero-rate. Zero-rate communication over unknown DMCs with
feedback was considered by (Tchamkerten Telatar, 2005), which provided an
upper bound on the error exponent. For $\ALPHABET Q_p$ this upper bound
evaluates to $E''_p = \tfrac 12 B_p$. 

Combine these two upper bounds to obtain a unified upper bound

$$E^*(γ) = B_p (1 - \max(\tfrac 12, γ_p )).$$

\begin{Schunk}
\begin{Sinput}
>   compound.upper <- function(p,R)
>   {
>     return (pmin(0.5*bsc.B(p), compound.burnashev(p,R)) )
>   }
\end{Sinput}
\end{Schunk}

The error exponent of our scheme for $γ_p = γ_{1-p}$ is given by 
$$E_p ≥ \frac{D(0.5 \| p)   D(p \| 1-p)}
            {D(0.5\|p) + D(p \| 1-p)} 
            (1-γ_p)                         $$

\begin{Schunk}
\begin{Sinput}
>   compound.training <- function(p,R) 
>   {
>     γ = R/bsc.C(p)
>     q = 0.5
>     a = binary.D(0.5, p)
>     b = binary.D(p, 1-p)
>     return (a*b*(1-γ)/(a+b))
>   }
\end{Sinput}
\end{Schunk}

These upper bounds, along with the error exponent of our scheme for
$γ_p = γ_{1-p}$, are plotted using the function below.


\begin{Schunk}
\begin{Sinput}
> # Upper bound on error exponent 
> compoundBsc.plotBound <- function (p=0.1)
> {
>   eps = 10e-3
>   R <- seq(eps, bsc.C(p) - eps, length=100)
> 
>   pdf("upper.pdf", width=7.5, height=4.5)
>   par (lty = "solid", lwd = 1, bty = "l")           # An "L" box
>   plot(R, compound.burnashev(p,R), 
>        type = "l"  ,
>        lty  = "44" ,
>        lwd  = 1    ,
>        xlab = "Rate" ,
>        ylab = "Error Exponent")
> 
>   abline (h = 0.5*bsc.B(p), lty = "22")
> 
>   lines (R, compound.upper(p,R), lty = "solid")
> 
>   lines (R, compound.training(p,R), lty="1343") 
>   legend("topright",
>          c("Burnashev exponent"          ,
>            "Zero-rate error exponent"    , 
>            "Combined upper bound"        ,
>            "Exponent of proposed scheme "),
>         lty = c("44", "22", "solid", "1343"),
>         cex = 0.8, lwd=1, bty="n")
> 
>   dev.off()
>   embedFonts("upper.pdf", options="-dPDFSETTINGS=/prepress")
> }
\end{Sinput}
\end{Schunk}


![exponents][exponents]

[values]: exponent "Values of $E_p$ and $E_{1-p}$ for different values of $γ_p$"
[plot]: threshold "A semi-log plot of $φ(q,0.1)$ for $0.1 < q < 0.9$"
[exponents]: upper "Upper bound on error exponents"
    
  
